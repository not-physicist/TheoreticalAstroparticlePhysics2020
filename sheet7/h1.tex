\section{Quickies}
\begin{enumerate}[label=\alph*)]
\item For heavier elements, BBN are unable to explain the abundance of it. Stellar and supernova nucleosynthesis are the two ways to explain this. Stellar nucleosynthesis explain how new heavy nuclei are produced in stars during stellar evolution. It explains the abundances of elements from carbon to iron that we see today. Stars like our sun act as thermonuclear furnaces in which H and He are fused into heavier nuclei by increasingly high temperatures as the composition of the core evolves.[Wiki] 

\textcolor{blue}{Supernova nucleosynthesis occurs in the energetic environment in supernovae, in which the elements between silicon and nickel are synthesized in quasiequilibrium[13] established during fast fusion that attaches by reciprocating balanced nuclear reactions to 28Si. [Wiki]}

\item The reactions crucial in the early universe can be categorized as follows:
\begin{enumerate}[label=\roman*)]
\item $p(n, \gamma)D$, production of deuterium, initial stage.
\item $D(p,\gamma) ^3 He$, $D(D, n) ^3 He$,$ D(D, p)T$,$ ^3 He(n, p)T$, preliminary reactions preparing material for $^4 He$ production.
\item $T(D, n) ^4 He$,$ ^3 He(D, p) ^4 He$, production of 4 He.
\item $T(\alpha, \gamma) ^7 Li$, $^3 He(\alpha, \gamma) ^7 Be$,$^ 7 Be(n, p) ^7Li$, production of the heaviest elements. $^7 Li(p, \alpha) ^4 He$, burning of $^7 Li$.
\end{enumerate}

The simplest nucleus that can be produced in the early universe is Deuterium ($^2 H$) via $p(n, \gamma)D$. This reaction would not take place if the temperature are high enough to overcome its binding energy and break D back to its constituents. Most of the light nuclei needs D to be produced. Hence, it is the production of deuterium that determines whether the nucleosynthesis will take place as well as the nucleosynthesis temperature. Thus, deuterium bottleneck.

\textcolor{blue}{However, since the binding energy of $^4 He$ is greater than that of deuterium, it is possible to to produce $^4 He$ without going through Deuterium. But, this does not happen because of something. I think not enough neutron or proton? or something bout temperature not high enough to produce 4He, but D...then since D abundant, D becomes dominating process?}

\end{enumerate}