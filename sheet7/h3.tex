\section{Neutron burning}
\begin{enumerate}[label=\alph*)]
\item Given $\langle\sigma \abs{\vec{v}} \rangle_{p(n,\gamma)D} \approx 6\times\SI{}{10^{-20} cm^3 s^{-1}}$ , $\eta_B = 6.2 \times 10^{-10}$
\begin{align}
\Gamma_{p(n,\gamma)D} &= n_p \cdot \langle\sigma \abs{\vec{v}} \rangle_{p(n,\gamma)D} \\
	&= \eta_B n_\gamma  \langle\sigma \abs{\vec{v}} \rangle_{p(n,\gamma)D} \nonumber \\ 
	&= \eta_B (g_\gamma \frac{\zeta(3)}{\pi^2}T^3)\langle\sigma \abs{\vec{v}} \rangle_{p(n,\gamma)D} \nonumber
\end{align}
Take $T_{NS} = \SI{80}{keV} =  80 \times 10^3 \times 8065\SI{}{cm^{-1}} $, $g_\gamma = 2$, $\zeta(3) \approx 1.2025$
\begin{align}
\Gamma_{p(n,\gamma)D} = \SI{0.002435}{s^{-1}} 
\end{align}
\textcolor{blue}{something felt wrong, I use the same equation with the values given in Gorubnov(pg191), I cant get the $0.5s^{-1}$ given in the book}\\
For a radiation dominated universe, taking $M^{\ast}_{pl} \approx M_{pl} \approx \SI{1.2209 e19}{GeV}$ . \textcolor{blue}{I dont know what $M_{Pl}$ to use.}
\begin{align}
H(T_{NS}) 	&= \frac{T^2}{M^{\ast}_{pl}} \nonumber \\
	&= \frac{(80\times \SI{e-6}{GeV})^2}{\SI{1.2209 e19}{GeV}} \nonumber  \\
	&= \SI{5.242e-28}{GeV}
\end{align}
$\Gamma_{p(n,\gamma)D} \gg H(T_{NS})$? dafuq with the units 
\item Since we are taking the assumption that the freezeout temperature is when the universe is radiation dominated, we use 
\begin{align}
H(T_{NS}) 	&= \frac{1}{2t} \\
\SI{5.242e-28}{GeV}	&=  \frac{1}{2t} \nonumber  \\
	t	&=  \frac{1}{2 \cdot (\SI{5.242e-28}{GeV})} \nonumber  \\
		&=   \SI{9.538e26}{GeV}\nonumber  
\end{align}
Something bout the the units . :|
\end{enumerate}

