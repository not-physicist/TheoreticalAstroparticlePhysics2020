\section{Neutron burning}
\begin{enumerate}[label=\alph*)]
\item Given $\langle\sigma \abs{\vec{v}} \rangle_{p(n,\gamma)D} \approx 6\times\SI{}{10^{-20} cm^3 s^{-1}}$ , $\eta_B = 6.2 \times 10^{-10}$
\begin{align}
\Gamma_{p(n,\gamma)D} &= n_p \cdot \langle\sigma \abs{\vec{v}} \rangle_{p(n,\gamma)D} \\
	&= \eta_B n_\gamma  \langle\sigma \abs{\vec{v}} \rangle_{p(n,\gamma)D} \nonumber \\ 
	&= \eta_B (g_\gamma \frac{\zeta(3)}{\pi^2}T^3)\langle\sigma \abs{\vec{v}} \rangle_{p(n,\gamma)D} \nonumber
\end{align}
During freeze out, $T = T_p$,
\begin{align}
H(T_p) &= \Gamma_{p(n,\gamma)D}  \\
\frac{T_p^2}{M^{\ast}_{pl}}  &= \eta_B (g_\gamma \frac{\zeta(3)}{\pi^2}T_p^3)\langle\sigma \abs{\vec{v}} \rangle_{p(n,\gamma)D} \nonumber \\
T_p &= \frac{\pi^2}{M^{\ast}_{pl} \eta_B (g_\gamma \zeta(3))\langle\sigma \abs{\vec{v}} \rangle_{p(n,\gamma)D}}
\end{align}
Assuming that the freeze out temperature is in the order of $\order{10keV}$ or smaller(which is well below the mass of electrons), there should be no relic electrons and positrons because all of them would annihilate to produce photons, as we learned from the our last assignment.
\begin{align}
g_* = 2 + \frac{7}{8} \cdot 4 +\frac{7}{8} \cdot 3 \cdot 2 \cdot \left( \frac{4}{11} \right)^{4/3} = 3.36
\end{align}
And with $g_\gamma = 2$, $\zeta(3) \approx 1.2025$, $M^{\ast}_{Pl} =\frac{M_{Pl}}{1.66\sqrt{g_*}}$, we have
\begin{align}
T_p &= \SI{0.316}{keV}
\end{align}
We can see that $T_p \ll T_{NS}$, which make sense because if the protons were to freeze out before $T_{NS}$, the relic protons wouldnt be in thermal equilibrium to produce Deuterium and consequently allowing BBN.

\item Knowing that the temperature during the matter-radiation equality is about $\SI{3}{eV}$, and from what we calcaluted, $T_p \gg \SI{3}{eV}$. the reaction freeze out is during the radiation dominated era. This is also why it make sense to use this as an assumption in the first part. 
\begin{align}
H(T_p) 	&= \frac{1}{2t_p} \\
\frac{T_p^2}{M^{\ast}_{pl}} &= \frac{1}{2t_p} \nonumber  \\
t_p &= \frac{M^{\ast}_{pl}} {2T_p^2}  \nonumber  \\
	&= \SI{1.299e7}{s}\nonumber  
\end{align}
\end{enumerate}

