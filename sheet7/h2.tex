\section{Dark Radiation during BBN}
\begin{enumerate}[label=\alph*)]
   \item We know the temperature of deuterium production 
      \begin{equation*}
         T_{NS} \approx \SI{80}{\kilo\eV}
      \end{equation*}
      In order to compute the age of the Universe, need $g_*$ first. Neutrinos decoupled at $\sim \SI{2}{\mega\eV}$ and their temeratures are $(4/11)^{1/3}$. Then
      \begin{equation}
         g_* = 2 + \frac{7}{8} \cdot 3 \cdot 2 \cdot \left( \frac{4}{11} \right)^{4/3} = 3.36
      \end{equation}
      As a result,
      \begin{equation}
         t_{NS} = \frac{M_{Pl}}{1.66\sqrt{g_*}} \frac{1}{2 T^2_{NS}} = \SI{203.3}{\s}
      \end{equation}
   \item \textcolor{blue}{For some reason}, we assume the electrons are still in thermal contact with the plasma (at least this assumption is self-consistent). Then 
      \begin{equation}
         g_* = 2 + \frac{7}{8} \cdot 4 + \frac{7}{8} \cdot 3 \cdot 2 \cdot = 10.75
      \end{equation}
      Thus
      \begin{align}
         \Gamma_n &\stackrel{!}{=} H \notag \\
         C_n G_F^2 T^5 &\stackrel{!}{=} T^2 / M_{Pl}^* \notag \\
         T_n^3 &= \left(M_{Pl}^* C_n G_F^2 \right)^{-1} \notag \\
         T_n &= \left(M_{Pl}^* C_n G_F^2 \right)^{-1/3} = \SI{1.4}{\mega\eV}
      \end{align}
   \item
      If there were yet-to-be-discovered light particle, which could be in thermal equilirum with photons in present era, the freeze-out temperature of neutron will also be different, since we have explicitely used $g_*$ in the previous calculation. 

      Specifically, $g_*$ would be larger and consequently $T_n \propto g_*^{1/6}$ will be higher and the neutron decoupling would have happened slightly earlier. Since at this temperature neutrons and protons are non-relativistic, they obey Boltzmann distribution $n_{n,p}(T) \propto T_n^{3/2} e^{-m/T}$. For the sake of simplicity, we just ignore the mass difference between protons and neutrons here.       

      In the expression of $\nnnp$, we see $t_{NS}$ and it depends on $g_*$ in the sense that $g_*$ grows, $t_{NS}$ decreases. We could not make further statement how exactly the fraction will change since the function expression is rather complicated. We need to calculate in next part anyway.

   \item 
      First express effective degrees of freedom in terms of $\Delta N_{\text{eff}}$
      \begin{equation}
         g_* = 2 + \frac{7}{8} \cdot (3 + \Delta N_\text{eff}) \cdot 2 \cdot \left( \frac{4}{11} \right)^{4/3}
      \end{equation}
      
      $g_*$ dependence enters not only $T_n$, but also $t_{NS}$. Thus
      \begin{align*}
         \dv{X_{\fHe}}{N_\eff} &= - \frac{2}{ \left(1 + \frac{n_p(T_{NS})}{n_n(T_{NS})} \right)^2  } \dv{\frac{n_p(T_{NS})}{n_n(T_{NS})}}{N_\eff} \\
                               &= - \frac{X_{\fHe}^2}{2} \left( \pdv{\npnn}{T_n} \dv{T_n}{N_\eff} + \pdv{\npnn}{t_{NS}}\dv{t_{NS}}{N_\eff} \right)
      \end{align*}
      Write $\dd$ into $\Delta$ and rearrange the equation, we have
      \begin{equation}
         \frac{\Delta X_\fHe}{X_\fHe} = - \frac{X_\fHe}{2} \left( \pdv{\npnn}{T_n} \dv{T_n}{N_\eff} + \pdv{\npnn}{t_{NS}}\dv{t_{NS}}{N_\eff} \right) \Delta N_\eff
      \end{equation}
      It contains four derivatives, thus it is more economical to calculate it with Mathematica. With $T_n \approx \SI{1.4}{\mega\eV}$, $t_{NS}\approx \SI{200}{\s}$, $X_\fHe \approx 45\% $, $\Delta N_\eff=1$, and $\mu_n = \mu_p$, we get 
      \begin{equation}
         \frac{\Delta X_\fHe}{X_\fHe} \approx 2.8\%
      \end{equation}
\end{enumerate}
