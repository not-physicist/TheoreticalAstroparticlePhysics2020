\section{Decoupling of electron- and muon-neutrino}
Here we have neglected the existence of the Z-boson, neutrino oscillations and loop process.
\begin{enumerate}[label=\alph*)]
   \item At temperature above the muon mass $(T \gtrsim m_\mu \approx \SI{105}{MeV})$ and below  QCD phase transition $(T \lesssim T_{QCD} \approx \SI{170}{MeV})$, \\
The only relevant processes are weak interacting process with exchange of W bosons. As we are at temperature above the muon mass, muon and anti-muon play a role. As shown below are the 'simplified' feynman diagrams. (Note that no fermion arrow were drawn, because it implies both possible directions. And also l and l' can be muon or electron, while $\nu$ and $\nu'$ can be muon- or electron-neutrinos.)

\begin{equation*}
   \feynmandiagram[horizontal=v1 to v2]{
      i1[particle=\(\nu\)] -- v1 -- i2[particle=\(l\)],
      v1 --[photon, edge label=\(W\)] v2,
      f1[particle=\(\nu'\)] -- v2 -- f2[particle=\(l'\)],
   };
   \feynmandiagram[vertical=v1 to v2]{
      i1[particle=\(\nu'\)] -- v1 -- f1[particle=\(l'\)],
      v1 --[photon, edge label=\(W\)] v2,
      i2[particle=\(\nu\)] -- v2 -- f2[particle=\(l\)],
   };
   \feynmandiagram[vertical=v1 to v2]{
      i1[particle=\(\nu'\)] -- v1 -- f1[particle=\(l'\)],
      v1 --[photon, edge label=\(W\)] v2,
      i2[particle=\(l\)] -- v2 -- f2[particle=\(\nu\)],
   };
   \feynmandiagram[vertical=v1 to v2]{
      i1[particle=\(l'\)] -- v1 -- f1[particle=\(\nu'\)],
      v1 --[photon, edge label=\(W\)] v2,
      i2[particle=\(\nu\)] -- v2 -- f2[particle=\(l\)],
   };
\end{equation*}

Other processes to think about:
\begin{enumerate}
\item Since it stated that we are working at $T \lesssim T_{QCD} \approx \SI{170}{MeV}$, one may think that pion decays to leptons and leptons neutrinos should contribute, since pion mass is around 134 to 139 MeV. But again. since it is stated that "hadronic degrees of freedom in the plasma is already decayed away", we are not considering it here.
\item There is also $ \nu \nu \rightarrow W W $ and  $ \nu W \rightarrow \nu W $ process. But since the $m_W$ is around $\SI{80}{GeV}$ (in the GeV range, way higher than $T$), it is highly surpressed.
\end{enumerate}

\item At temperature below the muon mass, the production of muons will be surpressed while existing muons will decay into electrons, electron-neutrinos and muon-neutrinos. The process possible left for the existing muon-neutrino is only $\nu_\mu-\mu$ annihilation which produce $\nu_e-e$. This is not the process that keeps muon-neutrinos in thermal equilibrium because the inverse of this process (muon-production) is highly surpressed. So, at this point, muon-neutrinos will be decoupled from the plasma. Note that this is true only if the existence of Z-boson, neutrino oscillation and loop process are neglected.
If plasma is all the possible particles that interact either directly/indirectly to the photon,
and the relativistic degree of freedom is the degrees of freedom of all these particles, then:
\begin{align}
g_{\ast} (T>m_{\mu}) &= 2_{\gamma} + \frac{7}{8} (4_e + 4_{\mu} +  2_{\nu_{\mu}} + 2_{\nu_{e}}) =  12.5 \\
g_{\ast} (T \ll m_{\mu}) &= 2_{\gamma} + \frac{7}{8} (4_e +  2_{\nu_{e}}) =  7.25 
\end{align}

\textcolor{blue}{We had a discussion on what is the proper defintion of the relativistic degree of freedom of the plasma and the defintion of a plasma.  We weren't really sure. So, here are some other possible answer if we were wrong above.
\begin{itemize}
\item  If plasma is only photons and existing  charged particles, and the relativistic degree of freedom is the degrees of freedom of all these particles, then:
\begin{align}
g_{\ast} (T>m_{\mu}) &= 2_{\gamma} + \frac{7}{8} (4_e + 4_{\mu}) =  9 \\
g_{\ast} (T \ll m_{\mu}) &= 2_{\gamma} + \frac{7}{8} (4_e ) =   5.5
\end{align}
\item  If plasma is only photons and existing charged particles, but the relativistic degree of freedom is the degrees of freedom of all relativistic particles, then:
\begin{align}
g_{\ast} (T>m_{\mu}) &= 2_{\gamma} + \frac{7}{8} (4_e + 4_{\mu} +  2_{\nu_{\mu}} + 2_{\nu_{e}} + 2_{\nu_{\tau}}) =  14.25\\
g_{\ast} (T \ll m_{\mu}) &= 2_{\gamma} + \frac{7}{8} (4_e + 2_{\nu_{\mu}} + 2_{\nu_{e}} + 2_{\nu_{\tau}}) =  10.75
\end{align}
\end{itemize}
}

\item When $\frac{\Gamma}{H} > 1$ the expansion rate of the universe is slower than that of the scattering rate, and as such, it is still possible for the particle to interact/scatter and as such still at thermal equilibrium with the plasma. \\

When $\frac{\Gamma}{H} < 1$ , the expansion rate of the universe is faster than that of the scattering rate, and as such, the particles are far apart and are unable to interact/scatter effectively anymore. \\

The particles are decoupled when 
\begin{align*}
\frac{\Gamma}{H} &= 1 \\
G_F^2 T_{dec}^5 &= \frac{T_{dec}^2}{M_{Pl}}\\
T_{dec} &= \left( \frac{1}{G_F^2 M_{Pl}}\right)^{\frac{1}{3}}\\
&= \left( \frac{1}{(\SI{1.166e-5}{GeV})^2 (\SI{1.22e19}{GeV})}\right)^{\frac{1}{3}}\\
&= \SI{0.0008447}{GeV} \\ &= \SI{0.8447}{MeV} 
\end{align*}

\item Knowing that,\\
\begin{align}
sa^3 &\propto g_\ast T^3 a^3 = \text{const}
\end{align}
One can compare the ratio of the degrees of freedom and temperature at different times
\begin{align}
g_{\ast,\alpha}T_\alpha^3 &= g_{\ast,\beta}T_\beta^3 
\end{align}
and one would end up with the following relation
\begin{equation}
T_\beta = \left(\frac{g_{\ast,\alpha}}{g_{\ast,\beta}}\right)^{1/3}T_\alpha
\end{equation}
After neutrino decoupling, the neutrino temperature remains the same despite the expanding of the universe. However, when the temperature goes below electron mass, electrons and positrons annihilate leading to a higher photon temperature. 
So to find the temperature of the decoupled electron-neutrinos, one simply has to compare when $T > m_e$ and $T < m_e$.

At $T > m_e$, $T = T_{\nu_e}$, $g_{\ast,\nu_e} = 2_\gamma + \frac{7}{8}4_e = 5.5 $\\
At $T < m_e$, $T= T_{\gamma}$, $g_{\ast,\gamma} = 2_\gamma $

Now,
\begin{align}
T_{\nu_e} &= \left(\frac{g_{\ast,\gamma}}{g_{\ast,\nu_e}}\right)^{1/3}T_\gamma \\
&= \left( \frac{4}{11} \right)^{1/3} \left(\SI{2.725}{K}\right) \\
&\approx \SI{1.95}{K}
\end{align}

\item To find the decoupled muon-neutrinos temperature, one simply follow the same procedure again with different relativistic effective numbers of degree of freedom. Taking $g_{\ast} (T>m_{\mu})=  12.25$ 
\begin{align}
T_{\nu_\mu} &= \left(\frac{2}{12.25}\right)^{1/3}\SI{2.725}{K}\\
&\approx \SI{1.479}{K}
\end{align}
Recall that for realtivistic particles that obey the Fermi-Dirac statistics, 
\begin{align}
n \propto T^3
\end{align}
and since $T_{\nu_e} > T_{\nu_\mu}$, one would expect the number density of electron-neutrinos to be higher.


\end{enumerate}
