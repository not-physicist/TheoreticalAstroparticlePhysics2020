\section{Quickies}
Well, One can consider two points each on opposite edge of the observable unierse. At some time in the early universe, both points would emit their last photon and it would travel to earth where we detect it as CMB. They should not be in causal contact since they are both outside of each other's lightcone. Yet, when we look at the CMB data, it shows that our observable  universe is nearly isotropic and  homogeneous. This implies that the entire observable universe must have been causally connected long enough to come into thermal equilibrium. This give rise to the horizon problem. 
\textcolor{blue}{I dont really get it, the universe have the same initial state, since we are treating the whole universe as a single body/system in the beginning of time, the "informations" is "stored" in the whole system and not just two points on the universe. So even if the two points are not in causal contact at some point in the early universe, they both are treated as a single system anyway in the beginning of the unierse anyway. I think the problem is just assuming differnt things at different theory yet using both theories to describe a single phenomena?}
