\section{Quickies}
Well, One can consider two points each on opposite edge of the observable unierse. At some time in the early universe, both points would emit their last photon and it would travel to earth where we detect it as CMB. They should not be in causal contact since they are both outside of each other's lightcone. Yet, when we look at the CMB data, it shows that our observable  universe is nearly isotropic and homogeneous. This implies that the entire observable universe must have been causally connected long enough to come into thermal equilibrium. This give rise to the horizon problem. 

Although different regions are causally disconnected, they obey the same laws of physics. So why would we still expect them to be different? Our best explanation would be because of probabilistic property of thermodynamics. There might be fluctuations generated and they get disconnected causally.
