	\section{Quickies}

\begin{enumerate}[label=(\alph*)]
	\item What is the definition of the cosmological redshift? \\
		Cosmological redshift is the redshift caused by the expansion of the universe itself. If a photon is emitted at time $t_i$ with a given wavelength, $\lambda_i$, then the observed wavelength of the photon at present time (with the cosmological redshift, z) is 
		\begin{equation}
			\lambda_0 = \lambda_i \frac{a_0}{a(t_i)} = \lambda_i \left[ 1 + z(t_i) \right]
		\end{equation}
where $a_0$ is the scale factor at present time and $a(t_i)$ is the scale factor at time $t_i$.

	\item Explain the notion of co-moving coordinates. \\
		Since in GR we learned that physics doesnt care what coordinate system we choose, we can choose one that is "natural"/easiest. 	In the comoving frame, the worldlines of particles at rest in this frame are geodesics, ie particles intially at rest will remain at rest. Particles that are initially moving with respect to this frame will eventually come to rest in it. The comoving coordinates assign constant spatial coordinate values to observers who perceive the universe as isotropic. The comoving time coordinate is just the proper time by an observer at rest in the comoving frame. It is also the elapsed time since the Big Bang according to a clock of a comoving observer and is a measure of cosmological time. Comoving distance is the distance between two points measured along a path defined at the present cosmological time. The comoving spatial coordinates tell where an event occurs while cosmological time tells when an event occurs.\\
Reference:\\
1.\hyperlink{https://en.wikipedia.org/wiki/Comoving_and_proper_distances}{https://en.wikipedia.org/wiki/Comoving\_and\_proper\_distances} \\
2.Kolb, E. W., \& Turner, M. S. (1981). The early universe. Nature, 294(5841)
\end{enumerate} 

