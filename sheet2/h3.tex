\section{Equation of state for an exotic scalar field}
\newcommand{\Lagr}{\mathcal{L}}
\begin{enumerate}[label=(\alph*)]
	\item 
	Using the identity,
\begin{equation}
	\frac{\delta(\det A)}{\det A}  = \trace{\left(\frac{\delta A}{A}\right)}
\end{equation}
	One can show that 
\begin{align}
	\frac{\delta g}{g} = g^{\mu\nu} \delta g_{\mu\nu} \notag \\
	\delta g = g g^{\mu\nu} \delta g_{\mu\nu} 
\end{align}
Then,
\begin{align}
	\delta \sqrt{-g} &= -\frac{1}{2 \sqrt{-g}} \delta g \notag \\
				&= -\frac{1}{2 \sqrt{-g}} g g^{\mu\nu} \delta g_{\mu\nu} \notag  \\ 
				&= -\frac{1}{2} \sqrt{-g} g^{\mu\nu} \delta g_{\mu\nu} 
\end{align}
Now, our Enery-momentum Tensor:
\begin{align}
	T_{\mu\nu} 	&=  \frac{2}{\sqrt{-g}} \frac{\delta(\sqrt{-g}\mathcal{L})}{ \delta g^{\mu\nu}} \notag \\
				&= \frac{2}{\sqrt{-g}} \left[ \frac{\delta(\sqrt{-g})}{\delta g^{\mu\nu}} \mathcal{L}+ \sqrt{-g}\frac{\delta\mathcal{L}}{\delta g^{\mu\nu}}  \right]  \notag\\
				&= \frac{2}{\sqrt{-g}} \left( -\frac{1}{2} \sqrt{-g} g_{\mu\nu} \right) \mathcal{L} + 2\frac{\delta \mathcal{L}}{\delta g^{\mu\nu}}  \notag \\
				&= -g_{\mu\nu} \mathcal{L} + 2\frac{\delta\mathcal{L}}{\delta g_{\mu\nu}}  
\end{align}
Given that the lagrangian is \\
\begin{align}
		\Lagr &= -V_0 \sqrt{1-g^{\alpha\beta} \partial_\alpha \phi \partial_\beta \phi } 
\end{align}
Then,
\begin{align}
\frac{\delta\mathcal{L}}{\delta g^{\mu\nu}} &= -V_0 \left(\frac{1}{2}\right) \left( \frac{-\delta^\alpha_\mu \delta^\beta_\nu \partial_\alpha \phi \partial_\beta \phi}{\sqrt{1-g^{\lambda\rho} \partial_\lambda\phi \partial_\rho\phi }} \right) \notag \\
									&= \frac{V_0}{2} \frac{\partial_\mu \phi \partial_\nu \phi}{\sqrt{1-g^{\alpha\beta} \partial_\alpha \phi \partial_\beta \phi }}
\end{align}

Coming back to our Enery-momentum Tensor:
\begin{align}
	T_{\mu\nu} 	&= -g_{\mu\nu} \mathcal{L} + \frac{V_0 \partial_\mu \phi \partial_\nu \phi}{\sqrt{1-g^{\alpha\beta} \partial_\alpha \phi \partial_\beta \phi }}   \notag \\
				&= V_0\frac{ g_{\mu\nu}(1-g^{\alpha\beta} \partial_\alpha \phi \partial_\beta \phi) + \partial_\mu \phi \partial_\nu \phi}{\sqrt{1-g^{\alpha\beta} \partial_\alpha \phi \partial_\beta \phi }}
\end{align}
\item
Since we are considering a spatially homogenous scalar field $\phi(t)$, 
\begin{align}
\partial_i \phi =0
\end{align}
Either we are using Minkowski or FLRW metric, $g_{00} = 1$ regardless, 
\begin{align}
	T_{\mu\nu}	&= V_0 \frac{ g_{\mu\nu} (1-g^{00} \partial_0 \phi \partial_0 \phi ) + \partial_\mu \phi \partial_\nu \phi }{\sqrt{1-g^{00} \partial_0 \phi \partial_0 \phi }} \notag \\
				&= V_0 \frac{ g_{\mu\nu} (1-(\partial_t \phi)^2) + \partial_\mu \phi \partial_\nu \phi }{\sqrt{1- (\partial_t \phi)^2}} 
\end{align}
Taking the 00-component,
\begin{align}
	T_{00}		&= V_0 \frac{1-(\partial_t \phi)^2 + (\partial_t \phi)^2}{\sqrt{1- (\partial_t \phi)^2}} \notag	\\
	\rho		&= \frac{V_0 }{\sqrt{1- (\partial_t \phi)^2}} 
\end{align}
Taking the ij-component,
\begin{align}
	T_{ij}	&= V_0 \frac{ g_{ij} (1-(\partial_t \phi)^2) + \partial_i \phi \partial_j \phi }{\sqrt{1- (\partial_t \phi)^2}} \notag \\
	-Pg_{ij}	&= V_0 g_{ij} \sqrt{1- (\partial_t \phi)^2} \notag \\
	P 		&= -V_0 \sqrt{1- (\partial_t \phi)^2} 
\end{align}
Looking at the equation of state,
\begin{align}
\omega 	&= \frac{P}{\rho}\\
		&=	\frac{-V_0 \sqrt{1- (\partial_t \phi)^2}}{\frac{V_0 }{\sqrt{1- (\partial_t \phi)^2}} } \notag \\
		&=	(\partial_t \phi)^2 -1
\end{align}
if $\partial_t \phi \ll 1$
\begin{align}
	\omega		&=	 -1	\\
	P 			&=	 -\rho 
\end{align}
For positive energy density, the pressure exerted by the scalar field is negative.
The equation of state for an exotic scalar field (when $\partial_t \phi \ll 1$) is similar to that of a vacuum in a flat space–time. 
%https://arxiv.org/pdf/1209.2754v1.pdf\\
\end{enumerate}

