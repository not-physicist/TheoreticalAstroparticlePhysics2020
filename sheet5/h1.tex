\section{Quickies}
\begin{enumerate}[label=\alph*)]
	\item Briefly explain why the temperature of about \SI{0.3}{eV}, at which the production of neutral hydrogen becomes thermodynamically favored, is less than one would naively expect from the binding energy of neutral hydrogen, which is \SI{13.6}{eV}.\\
	We should look at distribution of particles, different temperature leads to different tails? I dunno man. *insert poop emoji*
	
	\item The baryon asymmetry of our universe can be quantified with the time independent	ratio $\Delta_B = \frac{n_B - n_{\bar{B}} }{s} $, where $n_B$ ($n_{\bar{B}}$) is the number density of baryons (anti-baryons)	and s is the entropy density of the universe. It turns out that $\Delta_B \approx 0.14 \eta_B$, where $\eta_B \approx \SI{6.1e-10}{}$ is the baryon to photon r ratio inferred from measurements of the	CMB and nuclear abundances from BBN. What does the smallness of this number imply for the chemical potential of baryons $\mu_B$ at temperatures much larger than both µB and the baryon mass? \\
	
	The difference in number density (from the previous sheet) is,
	\begin{equation}
	\frac{\Delta n }{g} = \alpha \mu T^2 \text{ with } \alpha \begin{cases} 1/6 \text{ for fermions} \\ 1/3 \text{ for bosons} \end{cases}
	\end{equation}
	The entropy (taken from the text) is,
	\begin{equation}
	s = g \frac{2\pi^2}{45}T^3
	\end{equation}
	Given that Baryon is a fermion, and $\Delta_B = 0.14 \eta_B$ with $\eta_B \approx 6.1 \cross 10^{-10}$,
	\begin{align*}
	0.14 (6.1 \cdot 10^{-10} ) &= \frac{ \frac{1}{6} \mu_B T^2}{\frac{2\pi^2}{45}T^3} \\
	0.14 (6.1 \cdot 10^{-10}) \frac{4\pi^2}{15}T	&= \mu_B \\
	\mu_B &\approx 2.2476 \cdot 10^{-10} T
	\end{align*}
	At high temperature, non-neglible $\mu_B$?
\end{enumerate}