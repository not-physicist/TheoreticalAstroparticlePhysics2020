\section{Time of recombination from photon last scattering}
\begin{enumerate}[label=\alph*)]
   \item Assume the expansion was only cause by non-relativistic matter with $\Omega_m = 0.27$.  From previous homework, we know its energy density is proportional to its number density
   \begin{equation}
      \rho_M = m_M \cdot n_M
      \label{math:rho}
   \end{equation}

   Per definition the ratios of $\Omega$ is the same as the ratio of energy density 
      \begin{equation}
         m_M n_M = \frac{\Omega_M}{\Omega_B} m_p n_B^{tot}(T)
         \label{math:mn}
      \end{equation}
      and we say the baryonic matter in this epoch is simply protons or hydrogen atoms, thus the mass $m_B = m_p$.

      Again the baryonic matter density is directly related to photon density (time dependence of energy density is cancelled out)
      \begin{equation}
         n_B^{tot} = \eta_B \cdot n_\gamma(T) = \eta_B \frac{2\zeta(3)}{\pi^2} T^3
         \label{math:nB}
      \end{equation}

      Putting these three equations (\ref{math:rho}), (\ref{math:mn}), and (\ref{math:nB}) together
      \begin{equation}
         \rho_M = \frac{\Omega_M}{\Omega_B} m_p \eta_b \frac{2\zeta(3)}{\pi^2} T^3
      \end{equation}

      In matter-dominated epoch (the first equation is valid for all types of Universe with zero curvature), we have
      \begin{align}
         \rho &= \frac{3}{8\pi G} H^2 \label{math:rho2}\\
         H(t) &= \frac{2}{3t} \label{math:Ht}
      \end{align}

      Then the Hubble parameter as last scattering is
      \begin{align}
         H(T_r) &= \left[\frac{8\pi G}{3}  \frac{\Omega_M}{\Omega_B} m_p \eta_b \frac{2\zeta(3)}{\pi^2} T^3 \right]^{1/2} \notag \\
                &= \left[ \frac{8 \pi}{3} \frac{0.27}{0.046} 6.2\cdot 10^{-10} \frac{2 \cdot 1.20}{\pi^2} \frac{ \SI{938.3}{\mega \eV} \cdot (\SI{0.26}{\eV})^3}{(\SI{1.22e19}{\giga \eV})^2} \right]^{1/2} \notag \\
                &= \SI{2.87e-38}{\giga \eV} \notag \\
                &= \SI{1.435e-24}{\cm \tothe{-1}} \notag \\
                &= \SI{4.45}{\mega\parsec \tothe{-1}}
      \end{align}
      The time can be computed with (\ref{math:Ht})
      \begin{equation}
         t = \frac{2}{3 H} = \SI{2.32e37}{\giga \eV \tothe{-1}} = \SI{15.3e12}{\s} = \SI{4.85e5}{ \year}
      \end{equation}

      Alternatively, we can rewrite the Friedmann equation in terms of temperature.
      We know that the energy density of radiation $\rho_\text{rad} \sim a^{-4}$. Then we can write
      \begin{equation*}
         a = \frac{c}{T} \Rightarrow 
         \dd{a} = - \frac{a}{T} \dd{T}
      \end{equation*}
      Thus
      \begin{align*}
         H^2(t) &= \left( \frac{\dot{a}}{a} \right)^2 = H_0^2 \Omega_M \left(\frac{T}{T_0}\right)^3 \\
         H(T) &= \frac{1}{T} \frac{\dd{T}}{\dd{t}}  \\
         t(T) &= \int_T^\infty \dd{T'} \frac{\dd{T'}}{H(T')T'} \\
         t(T) &= \frac{1}{H_0 \sqrt{\Omega_M}} \left( \frac{T_0}{T} \right)^{3/2}
      \end{align*}
      Plug in the temperatures ($T_0 = \SI{2.7}{\kelvin} = \SI{2.3e-4}{\eV}$), it gives the same age of the Universe.


   \item
      Now we have matter and rediation contributions. The Friedmann equation becomes
      \begin{equation*}
         H^2 (t) = \frac{8\pi G}{3} \rho = \frac{8 \pi G}{3} \left( \rho_\text{rad} + \rho_M \right)
      \end{equation*}
      Energy density of radiation, considering all relativitstic species, follows
      \begin{equation*}
         \rho_\text{rad} = g_* \frac{\pi^2}{30} T^4
      \end{equation*}
      where $g_*$ is the effective degrees of freedom. Without exact knowledge of the Universe, in particular the relativistic particles, $g_{*}$ is hard to determine. Thus we don't really want to plug it into Friedmann equation. (Actually I tried it under assumption of SM, e.g.~$g_*=3.36$, considering three neutrinos and phtons, and apply the normal matter dominated era formula. We get eventually $t=\SI{4.2e5}{\year}$.)

      Following the same receipt as the alternative solution to a)
      \begin{align}
         t(T_r) &= \frac{1}{H_0} \int_{T_r}^\infty \frac{\dd{T}}{T} \left[ \Omega_M \left( \frac{T}{T_0} \right)^3 + \Omega_{rad} \left( \frac{T}{T_0} \right)^4 \right]^{-1/2}\\
                &= \SI{3.58e5}{\year}
      \end{align}
      
      \iffalse
      The Friedmann equation with both contributions reads
      \begin{align*}
         \left( \frac{\dot{a}}{a} \right)^2 &= \frac{8\pi G}{3} \rho_c \left[ \Omega_M \left( \frac{a_0}{a} \right)^3 + \Omega_\text{rad} \left( \frac{a_0}{a} \right)^4 \right]\\
         \dv{a}{t} &= H_0 \left( \Omega_M \frac{a_0^3}{a}  + \Omega_\text{rad} \frac{a_0^4}{a^2}  \right) ^{1/2} \\
         \int_0^{t_r} \dd{t} &= \frac{1}{H_0} \int_0^{a_r} \dd{a}  \left( \Omega_M \frac{a_0^3}{a}  + \Omega_\text{rad} \frac{a_0^4}{a^2}  \right) ^{-1/2}
      \end{align*}
      Do a simple substitution and the integral on RHS gets flipped because of the negative sign
      \begin{align*}
         \int_0^{t_r} \dd{t} &= \frac{1}{H_0} \int_{T_r}^{\infty} \frac{\dd{T}}{T}   \left( \Omega_M \frac{a_0^3}{a^3}  + \Omega_\text{rad} \frac{a_0^4}{a^4}  \right) ^{-1/2} \\
         t_r &=  \frac{1}{H_0} \int_{T_r}^{\infty} \frac{\dd{T}}{T}  \left( \Omega_M \frac{T^3}{T_0^3}  + \Omega_\text{rad} \frac{T^4}{T_0^4}  \right)^{-1/2}
      \end{align*}
      With Python and its package, the integral is nuemerically evaluated 
      \begin{equation}
         t_r = \SI{2.1e6}{\year}
      \end{equation}

      Thus the Hubble parameter as a function of temperature is
      \begin{align}
         H'(T_r)^2 &=  \frac{8 \pi G}{3} \left( g_* \frac{\pi^2}{30} T^4 + \frac{\Omega_M}{\Omega_B} m_p \eta_b \frac{2\zeta(3)}{\pi^2} T^3 \right) \\
                   &= \SI{9.62e-38}{\giga \eV} > H(T_r)
      \end{align}
      \fi

   \item The Universe transitioned from radiation dominated to matter dominated at $\SI{0.76}{\eV}$. Thus in order to get the correct time of last scattering, one needs to include radiation.
\end{enumerate}
