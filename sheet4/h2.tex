\section{Number densities of particles and antiparticles}
System that we are looking at: 
\begin{itemize}
\item Massless particles, $m=0$,
\item chemical potential for particle $\mu$, chemical potential for anti particle $-\mu$,
\item thermal bath with $T \gg \mu$
\item spin degrees of freedom, g
\item $E^2 = \abs{\vec{p}}^2$ (because of $m=0$)
\end{itemize}
\begin{enumerate}[label=(\alph*)]
\item First we look at the number density, from here on we use $p$ as $\abs{\vec{p}}$ 
\begin{align}
n 	&= g \int d^{3} p f( \vec{p} ) \\ \notag 
	&= g \int 	f(\vec{p}) d\Omega p^2 dp \notag \\
	&= 4\pi g \int 	f(p) p^2 dp \notag  \\ 
	&= 4\pi g \int 	\frac{1}{(2\pi)^3} \frac{1}{e^{\frac{E-\mu}{T}}\pm 1}p^2 dp \notag  \\ 
	&= \frac{g}{(2\pi^2)} \int \frac{1}{e^{\frac{p-\mu}{T}}\pm 1}p^2 dp \notag \\ 
n 	&= \frac{g}{(2\pi^2)} \int \left( \frac{1}{e^{\frac{p}{T}}\pm 1} + \frac{e^{\frac{p}{T}}}{(e^{\frac{p}{T}}\pm 1)^2}\frac{\mu}{T}\right)p^2 dp  
\end{align}
Now getting the equation from the question
\begin{align}
\frac{\Delta n}{g} 	&= \frac{n-\bar{n}}{g} \\ \notag
				&= \frac{1}{\pi^2} \int \left( \frac{e^{\frac{p}{T}}}{(e^{\frac{p}{T}}\pm 1)^2}\frac{\mu}{T}\right)p^2 dp \notag
\end{align}
\item  Now using $z=\frac{p}{T}$
\begin{align}\label{eq:ng}
\frac{\Delta n}{g}	&= \frac{1}{\pi^2} \int \left( \frac{e^{z}}{(e^{z}\pm 1)^2} \frac{\mu}{T} \right)(zT)^2 d(zT) \notag \\ 
				&= \frac{\mu T^2}{\pi^2} \int \frac{z^2 e^{z}}{(e^{z} \pm 1)^2} dz 
\end{align}
Focus on the integral, one can do integrate by parts
\begin{align*}
   &\int \dd{z} \frac{z^2 e^z}{(e^z \pm 1)^2} \\
  =& \int \dd{ \left(- \frac{1}{e^z \pm 1} \right)} z^2 \\
  =& \left. -\frac{z^2}{e^z \pm 1} \right|_{0}^{\infty} + 2\int \dd{z} \frac{z}{e^z \pm 1} 
\end{align*}
If one looks at the first term graphically, one would notice that the at the boundaries, the term is equals to zero. One can then use the integrals given in the question to evaluate the second term:
\begin{equation*}
  2\int \frac{\dd{z}z}{e^z \pm 1} = 
\begin{cases}
   2\cdot\frac{2 - 1}{2} \pi^2 \frac{1}{6} & \text{fermions} \\
   2\cdot\frac{(2\pi)^2}{4}\frac{1}{6} & \text{bosons}
\end{cases}
=
\begin{cases}
   \frac{\pi^2}{6} & \text{fermions} \\
   \frac{\pi^2}{3} & \text{bosons}
\end{cases}
\end{equation*}
Finally,
\begin{equation}
\frac{\Delta n}{g} 	= 
\begin{cases}
   \frac{\mu T^2}{6} & \text{fermions} \\
   \frac{\mu T^2}{3} & \text{bosons}
\end{cases}
\end{equation}

\end{enumerate}

