\section{Quickies}
\begin{enumerate}[label=(\alph*)]
\item Briefly argue why the chemical potential $\mu_{\gamma}$ of photons is zero.\\

We know that one can introduce chemical potential $\mu_i$ to conserved quantum numbers $Q^{(i)}$ to deal with all relevant reactions in thermalized system with various
particle species. The chemical potential for particle of type A would then be
\begin{equation} \label{eq:1}
\mu_A = \sum_i \mu_i Q^{(i)}_A
\end{equation}
where $Q^{(i)}_A$ are the quantum numbers carried by particle A.\\

And for a photon, all of its quantum numbers are 0, eg. charge neutral, no lepton/baryon numbers, color neutral, etc. As such one can easily deduce that $\mu_{\gamma} = 0$\\

Another way of looking at this is that we know photon is its own antiparticle. A consequence of the equation \ref{eq:1} is that $\mu_{\bar{A}} = -\mu_A $, which again leads to $\mu_{\gamma}=0$ %intuitively?
\item At which approximate temperature did our universe transition from being dominated by radiation to a phase, where its energy density was dominated by non-relativistic matter? What is the corresponding redshift?\\
To get the transition, one need to consider when the energy density of non-relativistic matter and the energy density of radiations are equal:
\begin{align}
\frac{8\pi}{3}G\rho_c \Omega_M \left( \frac{a_0}{a_{eq}} \right)^3 &= \frac{8\pi}{3}G\rho_c \Omega_{rad} \left( \frac{a_0}{a_{eq}} \right)^4 \\
\Omega_M &= \Omega_{rad} \left( \frac{a_0}{a_{eq}} \right) \notag \\
\frac{\Omega_M }{\Omega_{rad}} &= \frac{a_0}{a_{eq}} = z_{eq} +1 
\end{align}
Given that $\Omega_{rad} \approx 5.0\cdot 10^{-5}$ and $\Omega_M \approx 0.315$ 
\begin{align}
z_{eq} +1 &\approx 6300\\ 
T_{eq} =T_0(z_{eq} +1 ) &\approx 3\cdot(6300) K \approx 18900K
\end{align}

\end{enumerate}
